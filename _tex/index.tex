% Options for packages loaded elsewhere
\PassOptionsToPackage{unicode}{hyperref}
\PassOptionsToPackage{hyphens}{url}
\PassOptionsToPackage{dvipsnames,svgnames,x11names}{xcolor}
%
\documentclass[
  letterpaper,
  DIV=11,
  numbers=noendperiod]{scrartcl}

\usepackage{amsmath,amssymb}
\usepackage{iftex}
\ifPDFTeX
  \usepackage[T1]{fontenc}
  \usepackage[utf8]{inputenc}
  \usepackage{textcomp} % provide euro and other symbols
\else % if luatex or xetex
  \usepackage{unicode-math}
  \defaultfontfeatures{Scale=MatchLowercase}
  \defaultfontfeatures[\rmfamily]{Ligatures=TeX,Scale=1}
\fi
\usepackage{lmodern}
\ifPDFTeX\else  
    % xetex/luatex font selection
\fi
% Use upquote if available, for straight quotes in verbatim environments
\IfFileExists{upquote.sty}{\usepackage{upquote}}{}
\IfFileExists{microtype.sty}{% use microtype if available
  \usepackage[]{microtype}
  \UseMicrotypeSet[protrusion]{basicmath} % disable protrusion for tt fonts
}{}
\makeatletter
\@ifundefined{KOMAClassName}{% if non-KOMA class
  \IfFileExists{parskip.sty}{%
    \usepackage{parskip}
  }{% else
    \setlength{\parindent}{0pt}
    \setlength{\parskip}{6pt plus 2pt minus 1pt}}
}{% if KOMA class
  \KOMAoptions{parskip=half}}
\makeatother
\usepackage{xcolor}
\setlength{\emergencystretch}{3em} % prevent overfull lines
\setcounter{secnumdepth}{-\maxdimen} % remove section numbering
% Make \paragraph and \subparagraph free-standing
\makeatletter
\ifx\paragraph\undefined\else
  \let\oldparagraph\paragraph
  \renewcommand{\paragraph}{
    \@ifstar
      \xxxParagraphStar
      \xxxParagraphNoStar
  }
  \newcommand{\xxxParagraphStar}[1]{\oldparagraph*{#1}\mbox{}}
  \newcommand{\xxxParagraphNoStar}[1]{\oldparagraph{#1}\mbox{}}
\fi
\ifx\subparagraph\undefined\else
  \let\oldsubparagraph\subparagraph
  \renewcommand{\subparagraph}{
    \@ifstar
      \xxxSubParagraphStar
      \xxxSubParagraphNoStar
  }
  \newcommand{\xxxSubParagraphStar}[1]{\oldsubparagraph*{#1}\mbox{}}
  \newcommand{\xxxSubParagraphNoStar}[1]{\oldsubparagraph{#1}\mbox{}}
\fi
\makeatother


\providecommand{\tightlist}{%
  \setlength{\itemsep}{0pt}\setlength{\parskip}{0pt}}\usepackage{longtable,booktabs,array}
\usepackage{calc} % for calculating minipage widths
% Correct order of tables after \paragraph or \subparagraph
\usepackage{etoolbox}
\makeatletter
\patchcmd\longtable{\par}{\if@noskipsec\mbox{}\fi\par}{}{}
\makeatother
% Allow footnotes in longtable head/foot
\IfFileExists{footnotehyper.sty}{\usepackage{footnotehyper}}{\usepackage{footnote}}
\makesavenoteenv{longtable}
\usepackage{graphicx}
\makeatletter
\newsavebox\pandoc@box
\newcommand*\pandocbounded[1]{% scales image to fit in text height/width
  \sbox\pandoc@box{#1}%
  \Gscale@div\@tempa{\textheight}{\dimexpr\ht\pandoc@box+\dp\pandoc@box\relax}%
  \Gscale@div\@tempb{\linewidth}{\wd\pandoc@box}%
  \ifdim\@tempb\p@<\@tempa\p@\let\@tempa\@tempb\fi% select the smaller of both
  \ifdim\@tempa\p@<\p@\scalebox{\@tempa}{\usebox\pandoc@box}%
  \else\usebox{\pandoc@box}%
  \fi%
}
% Set default figure placement to htbp
\def\fps@figure{htbp}
\makeatother
% definitions for citeproc citations
\NewDocumentCommand\citeproctext{}{}
\NewDocumentCommand\citeproc{mm}{%
  \begingroup\def\citeproctext{#2}\cite{#1}\endgroup}
\makeatletter
 % allow citations to break across lines
 \let\@cite@ofmt\@firstofone
 % avoid brackets around text for \cite:
 \def\@biblabel#1{}
 \def\@cite#1#2{{#1\if@tempswa , #2\fi}}
\makeatother
\newlength{\cslhangindent}
\setlength{\cslhangindent}{1.5em}
\newlength{\csllabelwidth}
\setlength{\csllabelwidth}{3em}
\newenvironment{CSLReferences}[2] % #1 hanging-indent, #2 entry-spacing
 {\begin{list}{}{%
  \setlength{\itemindent}{0pt}
  \setlength{\leftmargin}{0pt}
  \setlength{\parsep}{0pt}
  % turn on hanging indent if param 1 is 1
  \ifodd #1
   \setlength{\leftmargin}{\cslhangindent}
   \setlength{\itemindent}{-1\cslhangindent}
  \fi
  % set entry spacing
  \setlength{\itemsep}{#2\baselineskip}}}
 {\end{list}}
\usepackage{calc}
\newcommand{\CSLBlock}[1]{\hfill\break\parbox[t]{\linewidth}{\strut\ignorespaces#1\strut}}
\newcommand{\CSLLeftMargin}[1]{\parbox[t]{\csllabelwidth}{\strut#1\strut}}
\newcommand{\CSLRightInline}[1]{\parbox[t]{\linewidth - \csllabelwidth}{\strut#1\strut}}
\newcommand{\CSLIndent}[1]{\hspace{\cslhangindent}#1}

\KOMAoption{captions}{tableheading}
\makeatletter
\@ifpackageloaded{caption}{}{\usepackage{caption}}
\AtBeginDocument{%
\ifdefined\contentsname
  \renewcommand*\contentsname{Table of contents}
\else
  \newcommand\contentsname{Table of contents}
\fi
\ifdefined\listfigurename
  \renewcommand*\listfigurename{List of Figures}
\else
  \newcommand\listfigurename{List of Figures}
\fi
\ifdefined\listtablename
  \renewcommand*\listtablename{List of Tables}
\else
  \newcommand\listtablename{List of Tables}
\fi
\ifdefined\figurename
  \renewcommand*\figurename{Figure}
\else
  \newcommand\figurename{Figure}
\fi
\ifdefined\tablename
  \renewcommand*\tablename{Table}
\else
  \newcommand\tablename{Table}
\fi
}
\@ifpackageloaded{float}{}{\usepackage{float}}
\floatstyle{ruled}
\@ifundefined{c@chapter}{\newfloat{codelisting}{h}{lop}}{\newfloat{codelisting}{h}{lop}[chapter]}
\floatname{codelisting}{Listing}
\newcommand*\listoflistings{\listof{codelisting}{List of Listings}}
\makeatother
\makeatletter
\makeatother
\makeatletter
\@ifpackageloaded{caption}{}{\usepackage{caption}}
\@ifpackageloaded{subcaption}{}{\usepackage{subcaption}}
\makeatother

\usepackage{bookmark}

\IfFileExists{xurl.sty}{\usepackage{xurl}}{} % add URL line breaks if available
\urlstyle{same} % disable monospaced font for URLs
\hypersetup{
  pdftitle={Master Thesis},
  pdfkeywords={Phosphorus release kinetics, Fertiizer requirements
prediction},
  colorlinks=true,
  linkcolor={blue},
  filecolor={Maroon},
  citecolor={Blue},
  urlcolor={Blue},
  pdfcreator={LaTeX via pandoc}}


\title{Master Thesis}
\author{Marc Pérez}
\date{}

\begin{document}
\maketitle
\begin{abstract}
Hier kommt das Abstract
\end{abstract}


\section{Preface}\label{preface}

\section{Abstract}\label{abstract}

(Knuth 1984)

\section{Introduction}\label{introduction}

\subsection{Complexity of Phosphorous}\label{complexity-of-phosphorous}

Phosphorous displays a wide range of behaviours in soils, in places
where organic, mineral and aqueous phases interface. In phases that
contain oxygen Phosphorous is almost exclusively present as several
derivates of Orthophosphate \(PO_4^{3-}\) It can be found as organic
molecules as anhydric- and ester-groups, being needed by all known
species as a constituent of DNA and energy transfer-processes. It can be
present as anorganic Phosphate either as mono-orthophosphate
\(PO_4^{3-}\) or poly-orthophosphate \(HO-(PO_2)_n-OH\), where it can
strongly interact with water, forming, depending on pH \(HPO_4^{2-}\) or
\(H_2PO_4^{-}\). The dissolved species of phosphate are subject to
adsorption to clay- and oxide-surfaces of the solid soil-phase, they
also form fallout-products such as Apatite, Vivianite etc. With the
present metal-cations in the solution. While the solubility constant of
most phosphate-salts are comparably low (Wert eingeben), meaning that
the fallout and formation of minerals happens at low chemical activities
of phosphate, phosphate often is leached from soil-surface-layers,
heavily reducing the efficacy of P-fertilization and presenting a
disturbance to P-limited ecosystems. Those phenomena, many of them being
physicochemically controlled, are influenced by parameters such as pH,
ionic-strength, clay-content, specific-surface of the solid phase,
amorphous \(Fe(OH)_3\)-content amorphous \(Al(OH)_3\)-content, in short
the phenomena depend heavily on the composition, distribution and
geometry of the soil. Those properties are considered to be stable
respectively long-term properties of a soil, when looked at it with the
interest of modelling the transport processes of Phosphate in soils.
Factors such as water-content, temperature, vegetation and precipitation
are factors that temporally can vary fast and to a certain degree
unpredictably. Organic forms of phosphates, prominently DNA or
oligonucleotides and phytate are also subject to physicochemical
reactions, mainly decomposition, but are foremost controlled in their
presence by enzymatic processes, where i.e.~plants form phytates in
seeds to provide the embryo a compact and specific reserve of phosphate,
but many bacteria possess via Phytases the ability to hydrolyse phytate
and use it for their own means. To assess and cover those phenomena,
models, dynamically describing the motion of Phosphorous in soils,
differentiate several pools of Phosphorous, most prominently the
organic-P, dissolved-P, adsorbed-P, mineral-P, where the difference in
temporal behaviour, such as the mean-reside-time invite a
differentiation between labile-P, semi-labile-P and so on.

\subsection{Principle Mechanism of P-transport to the
Plant}\label{principle-mechanism-of-p-transport-to-the-plant}

Plants that live in ground, interface with their roots the soil-matrix,
soil-solution and air-pockets. It is the interface between root and
soil-solution, where the exchange, particularly the uptake of water and
nutrients happens. High-affinity phosphate transporters are mainly
responsible for the transport of phosphate from the soil-solution into
the cytosol. The momentary concentration of orthophosphate
{[}\(H_nPO_4^{n-3}\){]} in the soil-solution is called the intensity. It
is influenced by sinks such as biological uptake, precipitation,
adsorption and leaching, sources are dissolution of phosphate-minerals,
mineralization resp. enzymatic hydrolization of organic Phosphate-esters
and -anhydrides. Since it is difficult and sometimes up to now
impossible to accurately assess all these factors, it has been shown
beneficial to use a more agnostic approach and regard the net-diffusion
rates, as a black-box model. For the sake of simplicity and feasibility,
all biological-uptake can be removed, such that only the interaction
between soil-matrix and soil-solution can be observed. Such a model
would consist of the soil-matrix, containing the adsorption-surfaces and
the interfacing soil-solution that exchanges phosphates until an
equilibrium is reached. There have been different approaches to capture
the diffusion of Phosphate. The IEK method employs \(^{31}P\) and
assesses the amount of P, that is exchangeable between both phases and
the temporal development of that amount. Flossmann \& Richter assessed
the net-diffusion rate and equilibrium-concentration, modelling the
diffusion-process as a first-order-kinetic. Whilst IEK observes the
diffusion in a steady-state, the approach of Flossmann \& Richter did
start with a system of dried fine soil and deionized water, and
subsequently sampling at different time-points, observing the
desorption-reaction-speed from start to equilibrium.

\subsection{Prediction of biological
sinks}\label{prediction-of-biological-sinks}

The prediction of biological sinks, particularly plant uptake is of
great practical and academic interest. To adequately estimate the
ability of a soil to provide sufficient \(Pi\) by assessing the status
of both soil and soil-solution is the basis for economically and
ecologically sustainable fertilization practice. Several approaches and
models exist and aim to connect estimations of the \(Pi\)-availability
and the expected requirements of crops. The predominant of models in use
try to measure the extractable \(Pi\) via different solvents, and
depending on agressivity of the solvent to identify different pools of
\(Pi\). The extraction of \(Pi\) with \(H_2O\) or weakly acidic aqueous
solutions are generally identified as the immediately available \(Pi\)
or the Intensity. Extractants with strong acids or chelators, able to
dissolve and desorb strongly bound \(Pi\) are identified as the
\(Pi\)-Pool that will eventually become available to plants and is often
called the Capacity. Such models continue to calculate the required
amount of \(Pi\) by starting with the expexted uptake of the planned
crop, this uptake is often

\section{Materials and Methods}\label{materials-and-methods}

\subsection{The Long-Term Phosphorus Fertilization
Experiment}\label{the-long-term-phosphorus-fertilization-experiment}

The soil samples for this thesis originate from a set of six long-term
field trials in Switzerland, established by Agroscope between 1989 and
1992. The primary objective of these experiments was to validate and
re-evaluate Swiss phosphorus (P) fertilization guidelines by assessing
long-term crop yield responses to varying P inputs across different
pedoclimatic conditions. A detailed description of the experimental
design and site characteristics can be found in Hirte et al.~(2021).

The experiment was set up as a \textbf{completely randomized block
design} with four field replications at each site. The core of the
experiment consists of six fixed-plot treatments representing different
P fertilization levels, which were applied annually as superphosphate
before tillage and sowing. These levels were based on percentages of the
officially recommended P inputs: 0\% (Zero), 33\% (Deficit), 67\%
(Reduced), 100\% (Norm), 133\% (Elevated), and 167\% (Surplus).

\subsection{Experimental Sites}\label{experimental-sites}

The six experimental sites are located in the main crop-growing regions
of Switzerland: \textbf{Rümlang-Altwi (ALT)}, \textbf{Cadenazzo (CAD)},
\textbf{Ellighausen (ELL)}, \textbf{Grabs (GRA)}, \textbf{Oensingen
(OEN)}, and \textbf{Zurich-Reckenholz (REC)}. The key soil properties
are summarized below.

\begin{longtable}[]{@{}
  >{\raggedright\arraybackslash}p{(\linewidth - 10\tabcolsep) * \real{0.0704}}
  >{\raggedright\arraybackslash}p{(\linewidth - 10\tabcolsep) * \real{0.3099}}
  >{\raggedleft\arraybackslash}p{(\linewidth - 10\tabcolsep) * \real{0.1268}}
  >{\raggedleft\arraybackslash}p{(\linewidth - 10\tabcolsep) * \real{0.1268}}
  >{\raggedleft\arraybackslash}p{(\linewidth - 10\tabcolsep) * \real{0.2394}}
  >{\raggedleft\arraybackslash}p{(\linewidth - 10\tabcolsep) * \real{0.1268}}@{}}

\caption{\label{tbl-sites-corrected}Soil characteristics of the six
long-term experimental sites. Data adapted from Hirte et al.~(2021).}

\tabularnewline

\toprule\noalign{}
\begin{minipage}[b]{\linewidth}\raggedright
Site
\end{minipage} & \begin{minipage}[b]{\linewidth}\raggedright
Soil Type (WRB)
\end{minipage} & \begin{minipage}[b]{\linewidth}\raggedleft
Clay (\%)
\end{minipage} & \begin{minipage}[b]{\linewidth}\raggedleft
Sand (\%)
\end{minipage} & \begin{minipage}[b]{\linewidth}\raggedleft
Organic C (g/kg)
\end{minipage} & \begin{minipage}[b]{\linewidth}\raggedleft
pH (H2O)
\end{minipage} \\
\midrule\noalign{}
\endhead
\bottomrule\noalign{}
\endlastfoot
ALT & Calcaric Cambisol & 22 & 48 & 21 & 7.9 \\
CAD & Eutric Fluvisol & 8 & 40 & 14 & 6.3 \\
ELL & Eutric Cambisol & 33 & 31 & 23 & 6.6 \\
GRA & Calcaric Fluvisol & 17 & 34 & 16 & 8.3 \\
OEN & Gleyic-calc. Cambisol & 37 & 32 & 24 & 7.1 \\
REC & Eutric Gleysol & 39 & 25 & 27 & 7.4 \\

\end{longtable}

\textsubscript{Source:
\href{https://Andrapodon.github.io/Master-Thesis-P-kinetics/index.qmd.html}{Article
Notebook}}

Soil samples for this thesis were collected on {[}Your Sampling Date{]}
from the topsoil layer (0-20 cm). {[}Add any further specific details
about your sampling strategy here{]}.

\begin{center}\rule{0.5\linewidth}{0.5pt}\end{center}

\subsection{Phosphorus Desorption
Kinetics}\label{phosphorus-desorption-kinetics}

The analysis of phosphorus (P) desorption kinetics was based on the
principles of sequential extraction established by Flossmann and Richter
(1982). The original method is described below, followed by the specific
protocol adapted for this study.

\subsubsection{Original Method of Flossmann and Richter
(1982)}\label{original-method-of-flossmann-and-richter-1982}

The foundational method aims to characterize the P replenishment
capacity of the soil. The procedure is as follows:

\begin{enumerate}
\def\labelenumi{\arabic{enumi}.}
\tightlist
\item
  \textbf{Removal of Soluble P}: 17.5 g of air-dried soil is shaken with
  350 ml of deionized water for one hour at 25 ± 1°C. The suspension is
  centrifuged and the supernatant is decanted to remove the readily
  soluble P fraction. This first extract is referred to as:

  \begin{itemize}
  \tightlist
  \item
    \textbf{Solution A}: Contains easily soluble P, which is discarded.
  \end{itemize}
\item
  \textbf{Kinetic Extraction}: The remaining soil pellet is resuspended
  with another 350 ml of deionized water. Subsamples of the suspension
  are taken at specific time intervals, yielding the following extracts
  for kinetic analysis:

  \begin{itemize}
  \tightlist
  \item
    \textbf{Solution B}: Subsample taken after \textbf{10 minutes}.
  \item
    \textbf{Solution C}: Subsample taken after \textbf{30 minutes}.
  \item
    \textbf{Solution D}: Subsample taken after \textbf{120 minutes}.
  \end{itemize}
\item
  \textbf{Analysis}: The P concentration in Solutions B, C, and D is
  determined colorimetrically using the molybdenum blue method according
  to Murphy and Riley (1962).
\end{enumerate}

\subsubsection{Adapted Kinetic Protocol for This
Study}\label{adapted-kinetic-protocol-for-this-study}

For this thesis, the original method was modified to capture the
desorption process with a higher temporal resolution and using a
different soil-to-solution ratio.

\begin{enumerate}
\def\labelenumi{\arabic{enumi}.}
\item
  \textbf{Soil Suspension}: 10 g of air-dried soil was suspended in 200
  ml of deionized water. Unlike the original protocol, a pre-washing
  step to remove soluble P was not performed, meaning the measured
  desorption includes both the release of readily soluble P and the
  subsequent replenishment from the solid phase.
\item
  \textbf{Kinetic Extraction}: The suspension was shaken continuously,
  and subsamples were taken at eight time points to generate a detailed
  kinetic curve. The resulting extracts were:

  \begin{itemize}
  \tightlist
  \item
    \textbf{Extract 1}: Subsample taken after \textbf{2 minutes}.
  \item
    \textbf{Extract 2}: Subsample taken after \textbf{4 minutes}.
  \item
    \textbf{Extract 3}: Subsample taken after \textbf{10 minutes}.
  \item
    \textbf{Extract 4}: Subsample taken after \textbf{15 minutes}.
  \item
    \textbf{Extract 5}: Subsample taken after \textbf{20 minutes}.
  \item
    \textbf{Extract 6}: Subsample taken after \textbf{30 minutes}.
  \item
    \textbf{Extract 7}: Subsample taken after \textbf{45 minutes}.
  \item
    \textbf{Extract 8}: Subsample taken after \textbf{60 minutes}.
  \end{itemize}
\item
  \textbf{Analysis}: Each subsample was immediately filtered. The
  concentration of orthophosphate in the filtered extracts was
  determined colorimetrically using the \textbf{malachite green method}.
\end{enumerate}

\begin{center}\rule{0.5\linewidth}{0.5pt}\end{center}

\subsection{Statistical Analysis}\label{statistical-analysis}

The statistical analysis aimed to derive the kinetic parameters of
phosphorus (P) desorption for each soil sample. A two-stage approach was
employed, starting with a linearized model and progressing to a more
robust non-linear mixed-effects model.

\subsubsection{Initial Approach: Linearized
Model}\label{initial-approach-linearized-model}

{[}cite\_start{]}Following the conceptual framework of Flossmann and
Richter (1982), a first attempt was made to linearize the first-order
kinetic model{[}cite: 2, 3{]}. The differential equation describing P
release was solved and rearranged to yield a linear relationship:

\[ \log\left(1 - \frac{P(t)}{P^S}\right) = -kt \]

In this formulation, \(P(t)\) is the concentration of desorbed P at time
\(t\), \(k\) is the rate constant, and \(P^S\) is the maximum desorbable
P, acting as the asymptote. {[}cite\_start{]}For this initial model, the
\(P^S\) parameter was not estimated from the kinetic data itself but was
calculated \emph{a priori} for each sample as the difference between the
phosphorus concentration measured by the Olsen method
(\(P_\text{Olsen}\)) and the initially water-soluble P concentration
(\(P_{H_2O}\)){[}cite: 3, 7{]}.

While this approach allowed for the use of simple linear regression to
estimate \(k\), the reliance on an externally calculated asymptote led
to poor model fits and physically unrealistic results for several
samples. Consequently, this method was discarded in favor of a more
direct modeling approach.

\subsubsection{Final Approach: Non-Linear Mixed-Effects
Model}\label{final-approach-non-linear-mixed-effects-model}

{[}cite\_start{]}To overcome the limitations of the linearized model, a
non-linear mixed-effects model (\texttt{nlme}) was implemented to
estimate both the rate constant \(k\) and the asymptote \(P^S\)
simultaneously and directly from the time-series data{[}cite: 29{]}.

The model was fitted to the exact solution of the first-order rate
equation:

\[ P(t) = P^S \times (1 - e^{-k \times t'}) \]

Here, \(t'\) represents the adjusted time. {[}cite\_start{]}Based on
preliminary analysis, it was observed that some P dissolves rapidly
before the first measurement point{[}cite: 25{]}. {[}cite\_start{]}To
account for this, the measured time was adjusted by adding a 3-minute
offset (\(t' = t_\text{min} + 3\)){[}cite: 7, 25{]}.

{[}cite\_start{]}The model was implemented in R using the \texttt{nlme}
package{[}cite: 29{]}. To account for the hierarchical structure of the
data, where multiple time points are nested within each unique soil
sample, the model was specified with both fixed and random effects. The
overall population means for \(P^S\) and \(k\) were estimated as
\textbf{fixed effects}. {[}cite\_start{]}To capture the unique
desorption characteristics of each individual soil sample (defined by
its unique ID, \texttt{uid}), subject-specific deviations from the fixed
effects for both \(P^S\) and \(k\) were modeled as \textbf{random
effects}{[}cite: 29{]}.

This approach allowed for the robust estimation of individual kinetic
parameters for each soil sample, providing the \texttt{k} and
\texttt{P\^{}S} coefficients used for all subsequent analyses in this
thesis.

\emph{{[}The subsequent part of your analysis, where these coefficients
are merged with the long-term trial data, would be described next.{]}}

\section{Results}\label{results}

\section{Discussion}\label{discussion}

\section{Conclusion}\label{conclusion}

\section{Acknowledgments}\label{acknowledgments}

\section{Legal Disclosure}\label{legal-disclosure}

\section*{References}\label{references}
\addcontentsline{toc}{section}{References}

\phantomsection\label{refs}
\begin{CSLReferences}{1}{0}
\bibitem[\citeproctext]{ref-knuth84}
Knuth, Donald E. 1984. {``Literate Programming.''} \emph{Comput. J.} 27
(2): 97--111. \url{https://doi.org/10.1093/comjnl/27.2.97}.

\end{CSLReferences}

\section{Appendix}\label{appendix}

\section{Supplements}\label{supplements}




\end{document}
