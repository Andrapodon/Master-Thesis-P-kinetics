% Options for packages loaded elsewhere
% Options for packages loaded elsewhere
\PassOptionsToPackage{unicode}{hyperref}
\PassOptionsToPackage{hyphens}{url}
\PassOptionsToPackage{dvipsnames,svgnames,x11names}{xcolor}
%
\documentclass[
  a4paper,
]{article}
\usepackage{xcolor}
\usepackage{amsmath,amssymb}
\setcounter{secnumdepth}{5}
\usepackage{iftex}
\ifPDFTeX
  \usepackage[T1]{fontenc}
  \usepackage[utf8]{inputenc}
  \usepackage{textcomp} % provide euro and other symbols
\else % if luatex or xetex
  \usepackage{unicode-math} % this also loads fontspec
  \defaultfontfeatures{Scale=MatchLowercase}
  \defaultfontfeatures[\rmfamily]{Ligatures=TeX,Scale=1}
\fi
\usepackage{lmodern}
\ifPDFTeX\else
  % xetex/luatex font selection
  \setmainfont[]{Latin Modern Roman}
  \setsansfont[]{Latin Modern Sans}
  \setmonofont[]{Latin Modern Mono}
\fi
% Use upquote if available, for straight quotes in verbatim environments
\IfFileExists{upquote.sty}{\usepackage{upquote}}{}
\IfFileExists{microtype.sty}{% use microtype if available
  \usepackage[]{microtype}
  \UseMicrotypeSet[protrusion]{basicmath} % disable protrusion for tt fonts
}{}
\makeatletter
\@ifundefined{KOMAClassName}{% if non-KOMA class
  \IfFileExists{parskip.sty}{%
    \usepackage{parskip}
  }{% else
    \setlength{\parindent}{0pt}
    \setlength{\parskip}{6pt plus 2pt minus 1pt}}
}{% if KOMA class
  \KOMAoptions{parskip=half}}
\makeatother
% Make \paragraph and \subparagraph free-standing
\makeatletter
\ifx\paragraph\undefined\else
  \let\oldparagraph\paragraph
  \renewcommand{\paragraph}{
    \@ifstar
      \xxxParagraphStar
      \xxxParagraphNoStar
  }
  \newcommand{\xxxParagraphStar}[1]{\oldparagraph*{#1}\mbox{}}
  \newcommand{\xxxParagraphNoStar}[1]{\oldparagraph{#1}\mbox{}}
\fi
\ifx\subparagraph\undefined\else
  \let\oldsubparagraph\subparagraph
  \renewcommand{\subparagraph}{
    \@ifstar
      \xxxSubParagraphStar
      \xxxSubParagraphNoStar
  }
  \newcommand{\xxxSubParagraphStar}[1]{\oldsubparagraph*{#1}\mbox{}}
  \newcommand{\xxxSubParagraphNoStar}[1]{\oldsubparagraph{#1}\mbox{}}
\fi
\makeatother


\usepackage{longtable,booktabs,array}
\usepackage{calc} % for calculating minipage widths
% Correct order of tables after \paragraph or \subparagraph
\usepackage{etoolbox}
\makeatletter
\patchcmd\longtable{\par}{\if@noskipsec\mbox{}\fi\par}{}{}
\makeatother
% Allow footnotes in longtable head/foot
\IfFileExists{footnotehyper.sty}{\usepackage{footnotehyper}}{\usepackage{footnote}}
\makesavenoteenv{longtable}
\usepackage{graphicx}
\makeatletter
\newsavebox\pandoc@box
\newcommand*\pandocbounded[1]{% scales image to fit in text height/width
  \sbox\pandoc@box{#1}%
  \Gscale@div\@tempa{\textheight}{\dimexpr\ht\pandoc@box+\dp\pandoc@box\relax}%
  \Gscale@div\@tempb{\linewidth}{\wd\pandoc@box}%
  \ifdim\@tempb\p@<\@tempa\p@\let\@tempa\@tempb\fi% select the smaller of both
  \ifdim\@tempa\p@<\p@\scalebox{\@tempa}{\usebox\pandoc@box}%
  \else\usebox{\pandoc@box}%
  \fi%
}
% Set default figure placement to htbp
\def\fps@figure{htbp}
\makeatother


% definitions for citeproc citations
\NewDocumentCommand\citeproctext{}{}
\NewDocumentCommand\citeproc{mm}{%
  \begingroup\def\citeproctext{#2}\cite{#1}\endgroup}
\makeatletter
 % allow citations to break across lines
 \let\@cite@ofmt\@firstofone
 % avoid brackets around text for \cite:
 \def\@biblabel#1{}
 \def\@cite#1#2{{#1\if@tempswa , #2\fi}}
\makeatother
\newlength{\cslhangindent}
\setlength{\cslhangindent}{1.5em}
\newlength{\csllabelwidth}
\setlength{\csllabelwidth}{3em}
\newenvironment{CSLReferences}[2] % #1 hanging-indent, #2 entry-spacing
 {\begin{list}{}{%
  \setlength{\itemindent}{0pt}
  \setlength{\leftmargin}{0pt}
  \setlength{\parsep}{0pt}
  % turn on hanging indent if param 1 is 1
  \ifodd #1
   \setlength{\leftmargin}{\cslhangindent}
   \setlength{\itemindent}{-1\cslhangindent}
  \fi
  % set entry spacing
  \setlength{\itemsep}{#2\baselineskip}}}
 {\end{list}}
\usepackage{calc}
\newcommand{\CSLBlock}[1]{\hfill\break\parbox[t]{\linewidth}{\strut\ignorespaces#1\strut}}
\newcommand{\CSLLeftMargin}[1]{\parbox[t]{\csllabelwidth}{\strut#1\strut}}
\newcommand{\CSLRightInline}[1]{\parbox[t]{\linewidth - \csllabelwidth}{\strut#1\strut}}
\newcommand{\CSLIndent}[1]{\hspace{\cslhangindent}#1}



\setlength{\emergencystretch}{3em} % prevent overfull lines

\providecommand{\tightlist}{%
  \setlength{\itemsep}{0pt}\setlength{\parskip}{0pt}}



 


\usepackage{booktabs}
\usepackage{longtable}
\usepackage{array}
\usepackage{multirow}
\usepackage{wrapfig}
\usepackage{float}
\usepackage{colortbl}
\usepackage{pdflscape}
\usepackage{tabu}
\usepackage{threeparttable}
\usepackage{threeparttablex}
\usepackage[normalem]{ulem}
\usepackage{makecell}
\usepackage{xcolor}
\usepackage{fancyhdr}
\usepackage[english,mt]{ethidsc}
% Define the style but don't activate it with \pagestyle{fancy} here
\fancypagestyle{fancy}{
  \fancyhf{}
  \fancyhead[L]{\nouppercase{\leftmark}}
  \fancyhead[R]{\thepage}
  \renewcommand{\headrulewidth}{0.4pt}
}
\studentA{Marc Jerónimo Pérez y Roper}
\ethidA{13-938-311}
\emailA{marcpe@ethz.ch}
\supervision{Prof. Dr. Emmanuel Frossard \\ Dr. Frank Liebisch}
\declaration
\infopage
\makeatletter
\@ifpackageloaded{caption}{}{\usepackage{caption}}
\AtBeginDocument{%
\ifdefined\contentsname
  \renewcommand*\contentsname{Table of contents}
\else
  \newcommand\contentsname{Table of contents}
\fi
\ifdefined\listfigurename
  \renewcommand*\listfigurename{List of Figures}
\else
  \newcommand\listfigurename{List of Figures}
\fi
\ifdefined\listtablename
  \renewcommand*\listtablename{List of Tables}
\else
  \newcommand\listtablename{List of Tables}
\fi
\ifdefined\figurename
  \renewcommand*\figurename{Figure}
\else
  \newcommand\figurename{Figure}
\fi
\ifdefined\tablename
  \renewcommand*\tablename{Table}
\else
  \newcommand\tablename{Table}
\fi
}
\@ifpackageloaded{float}{}{\usepackage{float}}
\floatstyle{ruled}
\@ifundefined{c@chapter}{\newfloat{codelisting}{h}{lop}}{\newfloat{codelisting}{h}{lop}[chapter]}
\floatname{codelisting}{Listing}
\newcommand*\listoflistings{\listof{codelisting}{List of Listings}}
\makeatother
\makeatletter
\makeatother
\makeatletter
\@ifpackageloaded{caption}{}{\usepackage{caption}}
\@ifpackageloaded{subcaption}{}{\usepackage{subcaption}}
\makeatother
\usepackage{bookmark}
\IfFileExists{xurl.sty}{\usepackage{xurl}}{} % add URL line breaks if available
\urlstyle{same}
\hypersetup{
  pdftitle={P-release kinetic as a predictor for P-availability in the STYCS Trials},
  colorlinks=true,
  linkcolor={blue},
  filecolor={Maroon},
  citecolor={Blue},
  urlcolor={Blue},
  pdfcreator={LaTeX via pandoc}}


\title{P-release kinetic as a predictor for P-availability in the STYCS
Trials}
\author{Marc Pérez}
\date{}
\begin{document}
\maketitle
\begin{abstract}
Hier kommt das Abstract
\end{abstract}

\tableofcontents
\cleardoublepage 
\pagestyle{fancy}
\pagenumbering{arabic}


\section{Abstract}\label{abstract}

\section{Introduction}\label{introduction}

\subsection{The Complexity of
Phosphorus}\label{the-complexity-of-phosphorus}

Phosphorus (P) is an essential macronutrient for all known life, forming
a critical part of DNA and energy-transfer molecules (Berg et al., 2019;
National Institutes of Health, Office of Dietary Supplements, 2023;
Nelson et al., 2021). In soils---where organic, mineral, and aqueous
phases interface---its behavior is complex. In the presence of oxygen, P
exists almost exclusively as orthophosphate (\(PO_4^{3-}\)) and its
protonated forms (\(HPO_4^{2-}\) or \(H_2PO_4^{-}\)), depending on the
soil pH (Brady \& Weil, 2016; Sparks, 2003). These dissolved phosphate
species are highly reactive; they are subject to adsorption onto the
surfaces of clays and oxides and can precipitate with cations like
calcium, iron, and aluminum to form minerals with low solubility (Bohn
et al., 2002; Hinsinger, 2001; Sposito, 2008). Consequently, while total
soil P concentrations can be substantial, often ranging from 200 to 3000
mg kg⁻¹, the concentration of orthophosphate in the soil solution---the
form directly acquired by plant roots---is typically minuscule, often in
the range of 0.001 to 1 mg L⁻¹. This vast difference between the total
phosphorus stock and the infinitesimally small plant-available pool
represents a central challenge for global agricultural productivity
(Brady \& Weil, 2016; Holford, 1997; Sposito, 2008). This creates a
profound agronomic and environmental dilemma: while the majority of
applied P fertilizer is rapidly immobilized in the soil and remains
unavailable to crops, the fraction that is lost from fields via runoff
and erosion becomes a potent environmental pollutant. This fugitive P is
a primary driver of eutrophication in freshwater ecosystems, which are
often naturally P-limited Sharpley et al. (2003).

\textbf{Soil organic matter (SOM) adds another layer of complexity to
these interactions.} Organic acids released during the decomposition of
SOM can compete with phosphate for the same adsorption sites on mineral
surfaces, which can increase P concentrations in the soil solution.
Furthermore, humic substances can form stable complexes with cations
like Al³⁺ and Fe³⁺, preventing them from precipitating phosphate and
thereby enhancing its availability (Gerke, 2010; Stevenson, 1994).

\subsection{From Static Measurements to Dynamic
Understanding}\label{from-static-measurements-to-dynamic-understanding}

To manage this challenge, soil testing methods were developed to
estimate plant-available phosphorus. These tests are designed to measure
two key components of P availability: the \textbf{intensity factor},
which is the concentration of P in the soil solution at a given moment,
and the \textbf{capacity factor}, which represents the pool of weakly
adsorbed P that can readily replenish the soil solution. Traditional
methods used in Switzerland and the surrounding DACH region, such as
extraction with CO₂-saturated water or ammonium acetate EDTA (AAE10),
are designed to estimate the size of this readily available P pool (the
capacity factor) (Forschungsanstalt für Agrarökologie und Landbau (FAL),
1996; Schofield, 1955; Verband Deutscher Landwirtschaftlicher
Untersuchungs- und Forschungsanstalten (VDLUFA), 2000).While these tests
are invaluable for basic fertility assessment, they do not capture the
dynamic nature of P supply. A crucial missing piece of information is
the rate at which P is replenished into the soil solution from the solid
phase after being taken up by plant roots. This replenishment rate, or
\textbf{``kinetic factor''}, is vital for sustaining crop growth,
especially during periods of high demand Frossard et al. (2000).

The importance of these dynamics is not a new concept. As early as 1982,
\textbf{Flossmann and Richter} argued that characterizing the kinetics
of P release was essential for refining fertilizer recommendations
beyond what static tests alone could offer (Flossmann \& Richter, 1982).
Modern research has reinforced this view, showing that fertilization
strategies based solely on maintaining a critical soil test P (STP)
concentration can be inefficient (McDowell \& Sharpley, 2001; Rowe et
al., 2016). In Switzerland, this has led to the accumulation of ``legacy
P'' in many agricultural soils, and understanding the release kinetics
of this legacy P is key to both improving nutrient efficiency and
protecting water quality (Hirte et al., 2018). Furthermore, critical STP
levels are not constant; they are influenced by pedoclimatic factors
like soil texture and temperature, making a ``one-size-fits-all''
approach to fertilization suboptimal (Bell et al., 2013; Hirte, Richner,
et al., 2021; Sims \& Sharpley, 2005).

\subsection{Objectives and Research
Questions}\label{objectives-and-research-questions}

An ideal set of parameters for phosphorus (P) management must move
beyond simple agronomic sufficiency to encompass both environmental
stewardship and the biophysical realities of nutrient acquisition by
plants. To be truly effective, such parameters must: (1) be sensitive to
changes in the soil P status resulting from fertilizer inputs and crop
removal (the P balance) (Johnston et al., 2001); (2) correlate with the
risk of P loss to the environment (P export) (Sharpley et al., 2000);
and (3), most critically, reflect the kinetic nature of P supply to
plant roots, which is governed by the slow diffusion of phosphate in the
soil solution (Kuang et al., 2012; Nye \& Tinker, 2000).

This thesis hypothesizes that kinetic parameters describing P
desorption, derived from a simple laboratory extraction, can serve as
effective predictors for agronomic outcomes. To test this, soils were
sourced from the long-term Swiss agricultural experiment STYCS (Hirte,
Stüssel, et al., 2021), which provides an ideal platform for this
research. The experiment's multi-decade history has established stable P
equilibria across a wide and deliberately created gradient of P
availability (from 0\% to 167\% of recommended fertilization). This
allows for robust modeling of crop responses. Furthermore, the trial
encompasses six sites with diverse pedoclimatic conditions, ensuring
that any findings have broad applicability across different Swiss
agricultural landscapes. This study employs a modified version of the
Flossmann \& Richter kinetic test to derive the desorption rate (k) and
the desorbable P pool (\(P_{desorb}\)). The performance of these new
kinetic parameters will be compared against standard STP methods
(\(P_{CO_2}\) and \(P_{AAE10}\)) by addressing the following research
questions:

\subsubsection{Research Questions and
Hypotheses}\label{research-questions-and-hypotheses}

\paragraph{Research Question 1: How well do standard soil test P (STP)
methods predict agronomic outcomes and how do they relate to fundamental
soil
properties?}\label{research-question-1-how-well-do-standard-soil-test-p-stp-methods-predict-agronomic-outcomes-and-how-do-they-relate-to-fundamental-soil-properties}

\begin{itemize}
\item
  \textbf{Hypothesis 1a (Agronomic Performance):} The standard STP
  methods (\(P_{CO_2}\) and \(P_{AAE10}\)), which measure the P
  \emph{capacity} (the size of the readily available pool), will show a
  significant correlation with the P-Balance, as this is directly
  influenced by P inputs (Johnston et al., 2001). However, they will be
  weak predictors of relative crop yield and P-uptake, as these
  agronomic outcomes are more dependent on the \emph{rate} of P supply
  throughout the growing season (Hirte, Stüssel, et al., 2021).
\item
  \textbf{Hypothesis 1b (Relation to Soil Properties):} The measured STP
  values will be positively correlated with soil clay and organic carbon
  content, reflecting the greater number of sorption sites in these
  soils (Brady \& Weil, 2016). Conversely, the P\_AAE10 measurement will
  be negatively correlated with soil pH, particularly in soils with a
  \(pH > 6.8\), due to the chelation of \(Ca^{2+}\) and \(Mg^{2+}\) by
  the EDTA in the extractant, which reduces its effectiveness
  (Forschungsanstalt für Agrarökologie und Landbau (FAL), 1996).
\end{itemize}

\paragraph{Research Question 2: Can P desorption kinetics be reliably
characterized for the diverse soils of the STYCS trial, and how do the
derived kinetic parameters relate to soil
properties?}\label{research-question-2-can-p-desorption-kinetics-be-reliably-characterized-for-the-diverse-soils-of-the-stycs-trial-and-how-do-the-derived-kinetic-parameters-relate-to-soil-properties}

\begin{itemize}
\item
  \textbf{Hypothesis 2a (Methodological Feasibility):} The P desorption
  process in the STYCS soils will follow a first-order kinetic model.
  However, the original linear estimation method proposed by Flossmann
  \& Richter (1982) may be inaccurate because it relies on a potentially
  overestimated desorbable P pool (\(P_{desorb}\)). A non-linear
  modeling approach will provide more robust and replicable estimates of
  both the desorption rate constant (\emph{k}) and the desorbable P pool
  (\(P_{desorb}\)) (Kuang et al., 2012).
\item
  \textbf{Hypothesis 2b (Relation to Soil Properties):} The kinetic
  parameters will be significantly influenced by soil composition. The
  desorbable P pool (\(P_{desorb}\)) is expected to correlate positively
  with clay and organic matter content, which provide sorption surfaces.
  The rate constant (\emph{k}) is expected to be influenced by pH, as
  the speciation of orthophosphate changes, affecting its interaction
  with mineral surfaces and its mobility (Sparks, 2003).
\end{itemize}

\paragraph{Research Question 3: Can kinetic parameters significantly
improve the prediction of agronomic outcomes compared to standard static
STP
methods?}\label{research-question-3-can-kinetic-parameters-significantly-improve-the-prediction-of-agronomic-outcomes-compared-to-standard-static-stp-methods}

\begin{itemize}
\tightlist
\item
  \textbf{Hypothesis 3 (Improved Predictive Power):} Because plant P
  uptake is fundamentally limited by the slow diffusion of phosphate to
  the root surface, a dynamic measure is required for accurate
  prediction (Nye \& Tinker, 2000). Therefore, a model incorporating the
  kinetic parameters (\emph{k} and \(P_{desorb}\)), which together
  describe the replenishment rate of the soil solution, will explain a
  significantly greater proportion of the variance in relative yield and
  P-uptake compared to models based solely on the static STP
  measurements (Fardeau et al., 1991; Frossard et al., 2000).
\end{itemize}

\section{Materials and Methods}\label{sec-materials-and-methods}

\subsection{The Long-Term Phosphorus Fertilization
Experiment}\label{sec-the-long-term-phosphorus-fertilization-experiment}

The soil samples for this thesis originate from a set of six long-term
field trials in Switzerland, established by Agroscope between 1989 and
1992. The primary objective of these experiments was to validate and
re-evaluate Swiss phosphorus (P) fertilization guidelines by assessing
long-term crop yield responses to varying P inputs across different
pedoclimatic conditions. A detailed description of the experimental
design and site characteristics can be found in Hirte, Richner, et al.
(2021).

The experiment was set up as a \textbf{completely randomized block
design} with four field replications at each site. The core of the
experiment consists of six fixed-plot treatments representing different
P fertilization levels, which were applied annually as superphosphate
before tillage and sowing. These levels were based on percentages of the
officially recommended P inputs: 0\% (Zero), 33\% (Deficit), 67\%
(Reduced), 100\% (Norm), 133\% (Elevated), and 167\% (Surplus).

\subsection{Experimental Sites}\label{sec-experimental-sites}

The six experimental sites are located in the main crop-growing regions
of Switzerland: \textbf{Rümlang-Altwi (ALT)}, \textbf{Cadenazzo (CAD)},
\textbf{Ellighausen (ELL)}, \textbf{Grabs (GRA)}, \textbf{Oensingen
(OEN)}, and \textbf{Zurich-Reckenholz (REC)}. The key soil properties
are summarized below.

\begin{longtable}[]{@{}
  >{\raggedright\arraybackslash}p{(\linewidth - 10\tabcolsep) * \real{0.0704}}
  >{\raggedright\arraybackslash}p{(\linewidth - 10\tabcolsep) * \real{0.3099}}
  >{\raggedleft\arraybackslash}p{(\linewidth - 10\tabcolsep) * \real{0.1268}}
  >{\raggedleft\arraybackslash}p{(\linewidth - 10\tabcolsep) * \real{0.1268}}
  >{\raggedleft\arraybackslash}p{(\linewidth - 10\tabcolsep) * \real{0.2394}}
  >{\raggedleft\arraybackslash}p{(\linewidth - 10\tabcolsep) * \real{0.1268}}@{}}

\caption{\label{tbl-sites-corrected}Soil characteristics of the six
long-term experimental sites. Data adapted from Hirte et al.~(2021).}

\tabularnewline

\toprule\noalign{}
\begin{minipage}[b]{\linewidth}\raggedright
Site
\end{minipage} & \begin{minipage}[b]{\linewidth}\raggedright
Soil Type (WRB)
\end{minipage} & \begin{minipage}[b]{\linewidth}\raggedleft
Clay (\%)
\end{minipage} & \begin{minipage}[b]{\linewidth}\raggedleft
Sand (\%)
\end{minipage} & \begin{minipage}[b]{\linewidth}\raggedleft
Organic C (g/kg)
\end{minipage} & \begin{minipage}[b]{\linewidth}\raggedleft
pH (H2O)
\end{minipage} \\
\midrule\noalign{}
\endhead
\bottomrule\noalign{}
\endlastfoot
ALT & Calcaric Cambisol & 22 & 48 & 21 & 7.9 \\
CAD & Eutric Fluvisol & 8 & 40 & 14 & 6.3 \\
ELL & Eutric Cambisol & 33 & 31 & 23 & 6.6 \\
GRA & Calcaric Fluvisol & 17 & 34 & 16 & 8.3 \\
OEN & Gleyic-calc. Cambisol & 37 & 32 & 24 & 7.1 \\
REC & Eutric Gleysol & 39 & 25 & 27 & 7.4 \\

\end{longtable}

Soil samples for this thesis were collected in the year 2022 from the
topsoil layer (0-20 cm).

\begin{center}\rule{0.5\linewidth}{0.5pt}\end{center}

\subsection{Phosphorus Desorption
Kinetics}\label{sec-phosphorus-desorption-kinetics}

The analysis of phosphorus (P) desorption kinetics was based on the
principles of sequential extraction established by (Flossmann \&
Richter, 1982). The original method is described below, followed by the
specific protocol adapted for this study.

\subsubsection{Original Method of Flossmann and Richter
(1982)}\label{original-method-of-flossmann-and-richter-1982}

The foundational method aims to characterize the P replenishment
capacity of the soil. The procedure is as follows:

\begin{enumerate}
\def\labelenumi{\arabic{enumi}.}
\tightlist
\item
  \textbf{Removal of Soluble P}: 17.5 g of air-dried soil is shaken with
  350 ml of deionized water for one hour at 120 Hz in a horizontal
  soil-shaker. The suspension is centrifuged at 4000 rpm for 15 minutes
  and the supernatant is decanted to remove the readily soluble P
  fraction.
\item
  \textbf{Kinetic Extraction}: The remaining soil pellet is resuspended
  with another 350 ml of deionized water. Subsamples of the suspension
  are taken at specific time intervals (e.g., 10, 30, and 120 minutes).
\item
  \textbf{Analysis}: The P concentration in the subsamples is determined
  colorimetrically.
\end{enumerate}

\subsubsection{Adapted Kinetic Protocol for This
Study}\label{sec-adapted-kinetic-protocol-for-this-study}

For this thesis, the original method was modified to capture the
desorption process with a higher temporal resolution.

\begin{enumerate}
\def\labelenumi{\arabic{enumi}.}
\item
  \textbf{Pre-washing to Remove Soluble P}: A pre-washing step was
  performed to remove the readily soluble P fraction. 10 g of air-dried
  soil was suspended in 200 ml of deionized water and shaken for 60
  minutes at 120 Hz. The suspension was then centrifuged for 15 minutes
  at 4000 rpm, and the supernatant containing the soluble P was
  discarded.
\item
  \textbf{Kinetic Extraction}: The remaining soil pellet was resuspended
  in 200 ml of fresh deionized water. The suspension was shaken
  continuously, and subsamples were taken at eight time points to
  generate a detailed kinetic curve: \textbf{2, 4, 10, 15, 20, 30, 45,
  and 60 minutes}.
\item
  \textbf{Analysis}: Each subsample was immediately filtered. The
  concentration of orthophosphate in the filtered extracts was
  determined colorimetrically using the \textbf{malachite green method}
  (Van Veldhoven \& Mannaerts, 1987).
\end{enumerate}

\begin{center}\rule{0.5\linewidth}{0.5pt}\end{center}

\subsection{Statistical Analysis}\label{sec-statistical-analysis}

\subsubsection{Software and Statistical
Environment}\label{software-and-statistical-environment}

All data processing, statistical modeling, and visualization were
conducted using the R programming language (v. 4.2.2) (R Core Team,
2022). The primary packages used for the analysis were: - \texttt{nlme}
(Pinheiro et al., 2022) for fitting the non-linear mixed-effects models
to the kinetic data. - \texttt{lme4} (Bates et al., 2015) and
\texttt{lmerTest} (Kuznetsova et al., 2017) for fitting and testing the
linear mixed-effects models for agronomic and soil property analyses. -
\texttt{mlr3} (Lang et al., 2019) for the systematic feature selection
and model validation workflow.

\subsubsection{Modeling of Desorption
Kinetics}\label{sec-modeling-of-desorption-kinetics}

To derive the kinetic parameters, a non-linear mixed-effects model was
implemented using the \texttt{nlme} package. This approach was chosen to
simultaneously estimate the rate constant (\emph{k}) and the maximum
desorbable P (\(P_{desorb}\)) for each soil sample. The model was fitted
to the exact solution of the first-order rate equation:

\[P(t) = P_{desorb} \times (1 - e^{-k \times t'})\]

Where \(P(t)\) is the P concentration at time \(t\), and \(t'\) is an
adjusted time (\(t_{min}\) + 3 min) to account for the rapid initial
dissolution of P that occurs before the first measurement. In this
mixed-effects framework, the overall mean values for \(P_{desorb}\) and
\emph{k} were modeled as \textbf{fixed effects}, while sample-specific
deviations from these fixed effects were modeled as \textbf{random
effects} to capture the unique desorption characteristics of each
individual soil sample.

\subsubsection{Comparative Modeling of Soil and Agronomic
Parameters}\label{comparative-modeling-of-soil-and-agronomic-parameters}

To test the hypotheses of this thesis, two distinct sets of linear
mixed-effects models were constructed.


\begin{longtable}[]{@{}
  >{\raggedright\arraybackslash}p{(\linewidth - 6\tabcolsep) * \real{0.0807}}
  >{\raggedright\arraybackslash}p{(\linewidth - 6\tabcolsep) * \real{0.1615}}
  >{\raggedright\arraybackslash}p{(\linewidth - 6\tabcolsep) * \real{0.0932}}
  >{\raggedright\arraybackslash}p{(\linewidth - 6\tabcolsep) * \real{0.6646}}@{}}

\caption{\label{tbl-variables}Description of variables used in the
agronomic and soil models.}

\tabularnewline

\toprule\noalign{}
\begin{minipage}[b]{\linewidth}\raggedright
Abbreviation
\end{minipage} & \begin{minipage}[b]{\linewidth}\raggedright
Full.Name
\end{minipage} & \begin{minipage}[b]{\linewidth}\raggedright
Unit
\end{minipage} & \begin{minipage}[b]{\linewidth}\raggedright
Description
\end{minipage} \\
\midrule\noalign{}
\endhead
\bottomrule\noalign{}
\endlastfoot
\(Y_{rel}\) & Relative Yield & unitless & Plot yield normalized by the
national mean yield for that year and crop. \\
\(Y_{norm}\) & Normalized Yield & unitless & Plot yield normalized by
the site-specific median yield of the highest P treatment for that year
and crop. \\
\(P_{up}\) & P Uptake & kg P ha⁻¹ & Total P removed by the harvested
crop biomass over a growing season. \\
\(P_{bal}\) & P Balance & kg P ha⁻¹ & Net P budget, calculated as P
inputs (fertilizer) minus P outputs (uptake). \\
\(k\) & Rate Constant & min⁻¹ & First-order rate constant of P
desorption, representing the speed of P release. \\
\(P_{desorb}\) & Desorbable P & mg P L⁻¹ & Maximum desorbable P,
representing the size of the readily available P pool. \\
\(J_0\) & Initial P Flux & mg P L⁻¹ min⁻¹ & Product of \(k\) and
\(P_{desorb}\), representing the initial flux of P from the soil. \\
\(P_{CO2}\) & Water-Soluble P & mg P kg⁻¹ & Plant-available P measured
by CO₂-saturated water extraction
(Forschungsanstalt für Agrarökologie und Landbau (FAL), 1996). \\
\(P_{AAE10}\) & Chelate-Extractable P & mg P kg⁻¹ & Plant-available P
measured by the ammonium-acetate-EDTA extraction method
(Forschungsanstalt für Agrarökologie und Landbau (FAL), 1996). \\
\(Al_d\) & Dithionite-Extractable Al & mg Al kg⁻¹ & Free Al oxides
(crystalline and amorphous) extracted with
dithionite-citrate-bicarbonate
(Mehra \& Jackson, 1960). \\
\(Fe_d\) & Dithionite-Extractable Fe & mg Fe kg⁻¹ & Free Fe oxides
(crystalline and amorphous) extracted with
dithionite-citrate-bicarbonate
(Mehra \& Jackson, 1960). \\

\end{longtable}

\subsubsection{Model Assumptions and
Diagnostics}\label{model-assumptions-and-diagnostics}

The validity of the linear mixed-effects models (\texttt{lmer}) relies
on several key assumptions, primarily that the model residuals are
normally distributed and homoscedastic (i.e., have constant variance
across the range of predicted values). Prior to finalizing the models,
these assumptions were rigorously checked through visual inspection of
diagnostic plots, such as quantile-quantile (Q-Q) plots of the residuals
and plots of residuals versus fitted values.

Initial exploratory data analysis revealed that several of the predictor
variables, most notably the desorbable P pool (\(P_{desorb}\) or PS),
were strongly right-skewed. Using such variables directly in the linear
models would violate the assumptions of linearity and homoscedasticity,
leading to potentially biased and unreliable coefficient estimates.

To address this, various transformations (including square root and
logarithmic) were tested on the skewed variables. The natural
log-transformation (\texttt{log()}) was found to be the most effective
at normalizing the distribution of \texttt{PS} and producing
well-behaved model residuals that more closely met the required
assumptions. Therefore, \texttt{log(PS)} was used as a fixed effect in
all subsequent agronomic models to ensure the statistical validity of
the results.

\paragraph{Models of P Availability Metrics as a Function of Soil
Properties}\label{models-of-p-availability-metrics-as-a-function-of-soil-properties}

First, to investigate the underlying soil-chemical drivers of the
different P availability metrics (both kinetic and static), a series of
models was built to predict each metric from key soil properties.

\begin{itemize}
\tightlist
\item
  \textbf{Fixed Effects Structure:} The fixed effects were identical for
  all models in this category and included the primary soil physical and
  chemical properties: clay content, silt content, pH, organic carbon
  content, and dithionite-extractable iron (\(Fe_d\)) and aluminum
  (\(Al_d\)).
\item
  \textbf{Random Effects Structure:} The random effects structure
  accounted for the nested design of the STYCS experiment, with random
  intercepts for \texttt{year}, \texttt{Site}, \texttt{Site:block}, and
  \texttt{Site:Treatment}.
\end{itemize}

\paragraph{Comparative Models of Agronomic
Outcomes}\label{comparative-models-of-agronomic-outcomes}

Second, the predictive power of the kinetic parameters was directly
compared against that of the standard STP methods (``GRUD'' system). For
each agronomic response variable (Normalized Yield, P Uptake, and P
Balance), two competing models were built. These models shared an
identical random effects structure to ensure a fair comparison,
differing only in their fixed effects.

\begin{itemize}
\item
  \textbf{Random Effects Structure (for all agronomic models):} The
  structure
  \texttt{(1\textbar{}year)\ +\ (1\textbar{}Site)\ +\ (1\textbar{}Site:block)}
  was used to control for variations due to the growing season,
  location, and in-field spatial differences.
\item
  \textbf{Model 1: The Kinetic Approach}

  \begin{itemize}
  \tightlist
  \item
    \textbf{Fixed Effects:} This model used the kinetic parameters and
    their interaction: \texttt{k\ *\ log(PS)}. The interaction term
    tests the hypothesis that the benefit of a large desorbable P pool
    (\texttt{PS}) depends on the \emph{rate} (\texttt{k}) at which it
    can be accessed.
  \end{itemize}
\item
  \textbf{Model 2: The Standard STP (GRUD) Approach}

  \begin{itemize}
  \tightlist
  \item
    \textbf{Fixed Effects:} This model used the two standard Swiss soil
    tests and their interaction: \texttt{P\_CO2\ *\ P\_AAE10}.
  \end{itemize}
\end{itemize}

The relative performance of these two sets of models was then evaluated
to determine whether the kinetic parameters provided a significant
improvement in predictive power for key agronomic outcomes. Further the
relative performance of these two approaches, along with other
combinations of predictor sets, was rigorously evaluated using a machine
learning benchmark workflow implemented in the \texttt{mlr3} package
(Lang et al., 2019). The predictive power of each predictor set was
quantified using \textbf{5-fold cross-validation}. Performance was
measured as the percentage of explained variance on the hold-out data (1
- MSE / Var(y)), providing a robust and unbiased estimate of how well
each model would generalize to new data. This benchmark allowed for a
direct comparison of the information content provided by the kinetic
parameters versus the standard soil tests for predicting key agronomic
outcomes.

\section{Results}\label{results}

The results of this study are presented in two main parts. First, the
development and validation of the phosphorus (P) desorption kinetic
model are detailed, justifying the final modeling approach. Second, the
descriptive trends of both agronomic outcomes and soil P parameters in
response to long-term fertilization and site differences are explored
visually. Finally, the predictive power of the kinetic and standard P
parameters is formally evaluated using linear mixed-effects models.

\subsection{Establishment of the P-Desorption Kinetic
Model}\label{establishment-of-the-p-desorption-kinetic-model}

The primary goal was to derive two key parameters for each soil sample:
the desorbable P pool (\(P_{desorb}\)) and the rate constant (\(k\)).
The analysis proceeded in two stages: an initial test of a linearized
model, followed by the implementation of a more robust non-linear model.

\subsubsection{Initial Approach: Failure of the Linearized
Model}\label{initial-approach-failure-of-the-linearized-model}

Following the conceptual framework of Flossmann and Richter (1982), the
first-order kinetic equation was linearized. A core assumption of this
model is that the linear relationship must pass through the origin. To
test this, linear models were fitted to the transformed data for each
sample individually. The results revealed a systematic failure of this
assumption, as the estimated intercepts for the majority of samples were
highly significantly different from zero (p \textless{} 0.05). This
consistent statistical deviation indicated that the linearized approach
was not a valid representation of the data. The visual evidence in
Figure~\ref{fig-linearized-model} supports this conclusion.

\phantomsection\label{cell-fig-linearized-model}
\begin{figure}[H]

\centering{

\pandocbounded{\includegraphics[keepaspectratio]{index_files/figure-pdf/fig-linearized-model-1.pdf}}

}

\caption{\label{fig-linearized-model}Test of the linearized first-order
kinetic model. The plot visually supports the statistical finding that
many intercepts are not zero.}

\end{figure}%

\subsubsection{Final Approach: Successful Non-Linear
Model}\label{final-approach-successful-non-linear-model}

Given the statistical failure of the linearized model, a direct
non-linear modeling approach was adopted to estimate both \(P_{desorb}\)
and \(k\) simultaneously from the untransformed data. This approach does
not rely on the assumption of a zero intercept and proved to be far more
successful, accurately capturing the curvilinear shape of the desorption
data for nearly all samples (fig-nonlinear-model). The final parameters
were extracted from a non-linear mixed-effects model (\texttt{nlme}) to
account for the hierarchical data structure. \textbf{These final
\texttt{nlme}-derived coefficients were used for all subsequent
analyses.}

\phantomsection\label{cell-fig-nonlinear-model}
\begin{figure}[H]

\centering{

\pandocbounded{\includegraphics[keepaspectratio]{index_files/figure-pdf/fig-nonlinear-model-1.pdf}}

}

\caption{\label{fig-nonlinear-model}Non-linear first-order kinetic model
fits for P desorption over time. Points represent measured data and
solid lines represent the fitted model for each replicate.}

\end{figure}%

\subsection{Comparison with Isotopic Exchange Kinetics (IEK)
\{\#sec-comparison-with-isotopic-exchange-kinetics-(iek\}}\label{comparison-with-isotopic-exchange-kinetics-iek-sec-comparison-with-isotopic-exchange-kinetics-iek}

To validate the newly derived kinetic parameters against an established
benchmark, the capacity (\(P_{desorb}\)) and kinetic (\(k\)) parameters
were compared to data from Isotopic Exchange Kinetics (IEK) studies
previously conducted on the same long-term trial sites by Demaria et
al.~(2013). This comparison aims to determine if the simpler,
non-equilibrium desorption method used in this thesis captures similar
aspects of soil P dynamics as the more complex, equilibrium-based IEK
method.

The size of the desorbable P pool (\(P_{desorb}\)) was compared against
the long-term isotopically exchangeable P pool measured after 7 days
(\(E_{7d}\)). The desorption rate constant (\(k\)) was compared against
the IEK kinetic parameter measured after 24 hours (\(n_{1d}\)).
Spearman's rank correlation was used to robustly test for monotonic
trends between the different methods.

\begin{figure}

\begin{minipage}{0.50\linewidth}

\centering{

\pandocbounded{\includegraphics[keepaspectratio]{index_files/figure-pdf/fig-iek-comparison-1.pdf}}

}

\subcaption{\label{fig-iek-comparison-1}Capacity: \(P_{\text{desorb}}\)
vs \(E_{\text{7d}}\)}

\end{minipage}%
%
\begin{minipage}{0.50\linewidth}

\centering{

\pandocbounded{\includegraphics[keepaspectratio]{index_files/figure-pdf/fig-iek-comparison-2.pdf}}

}

\subcaption{\label{fig-iek-comparison-2}Kinetics: k vs
\(n_{\text{1d}}\)}

\end{minipage}%

\caption{\label{fig-iek-comparison}Correlation between
desorption-derived kinetic parameters and IEK-derived parameters. (A)
Capacity parameters: Desorbable P (\(P_{\text{desorb}}\))
vs.~Isotopically Exchangeable P (\(E_{\text{7d}}\)). (B) Kinetic
parameters: Rate Constant (\(k\)) vs.~IEK kinetic parameter
(\(n_{\text{1d}}\)).}

\end{figure}%

The analysis revealed a statistically significant, moderate positive
correlation between the capacity parameters, \(P_{desorb}\) and
\(E_{7d}\) (fig-iek-comparison). The Spearman's rank correlation
coefficient was 0.4 with a p-value of \textless{} 0.001.

Similarly, a statistically significant, moderate positive correlation
was found between the kinetic parameters, \(k\) and \(n_{1d}\)
(fig-iek-comparison). The Spearman's rank correlation coefficient was
0.36 with a p-value of \textless{} 0.001.

These results indicate that the simpler, non-equilibrium desorption
method used in this study successfully captures both the capacity and
intensity aspects of soil P lability, providing results that are
consistent with the more complex, equilibrium-based IEK method reported
by Demaria et al.~(2013).

\subsection{Effects of Fertilization on Agronomic and Soil
Parameters}\label{sec-effects-of-fertilization-on-agronomic-and-soil-parameters}

Having established a robust method to determine the kinetic parameters,
the next step was to explore the effects of the long-term P
fertilization treatments on both the agronomic outcomes and the soil P
test parameters.

\subsubsection{Agronomic Responses to P
Fertilization}\label{sec-agronomic-responses-to-p-fertilization}

The long-term application of different P fertilization levels had a
pronounced impact on the primary agronomic outcomes, including two
different metrics for yield, P Uptake (\(P_{up}\)), and P Balance
(\(P_{bal}\)), though the response varied considerably between sites
(Figure~\ref{fig-agronomic-responses}).

\phantomsection\label{cell-fig-agronomic-responses}
\begin{figure}[H]

\centering{

\pandocbounded{\includegraphics[keepaspectratio]{index_files/figure-pdf/fig-agronomic-responses-1.pdf}}

}

\caption{\label{fig-agronomic-responses}Agronomic response variables
across six P fertilization treatments and six experimental sites. Data
from 2017-2022.}

\end{figure}%

\textbf{Yield Metrics (}\(Y_{norm}\) and \(Y_{rel}\)): Both yield
metrics showed a generally positive response to P fertilization. The
site-normalized yield (\(Y_{norm}\)) shows the response relative to the
site's potential for that year, with most yields plateauing around the
Norm (100\%) treatment. The national-normalized yield (\(Y_{rel}\))
provides a broader context, showing how yields at each site compare to
the national average.

\textbf{P Uptake (}\(P_{up}\)): P uptake by crops followed a similar
trend to yield, increasing with fertilization, often continuing to
increase at the highest fertilization levels, suggesting luxury
consumption.

\textbf{P Balance (}\(P_{bal}\)): The P balance showed a strong, linear
relationship with fertilization. The Zero and Deficit treatments
resulted in a negative balance (mining soil P), while the Elevated and
Surplus treatments led to a significant P surplus.

\subsubsection{Soil P Parameters as a Function of P
Fertilization}\label{sec-soil-p-parameters-as-a-function-of-p-fertilization}

The different soil P test parameters, including the standard STP methods
and the newly derived kinetic parameters, all responded to the long-term
fertilization treatments (Figure~\ref{fig-soil-parameters}).

\phantomsection\label{cell-fig-soil-parameters}
\begin{figure}[H]

\centering{

\pandocbounded{\includegraphics[keepaspectratio]{index_files/figure-pdf/fig-soil-parameters-1.pdf}}

}

\caption{\label{fig-soil-parameters}Soil P parameters across six P
fertilization treatments and six experimental sites.}

\end{figure}%

\textbf{Standard STPs (}\(P_{CO_2}\) and \(P_{AAE10}\)): Both standard
soil P tests showed a clear and consistent increase with rising P
fertilization levels across all sites, confirming their sensitivity to
management.

\textbf{Kinetic Parameters (}\(k\), \(P_{desorb}\), and \(J_0\)): *
\textbf{Desorbable P (}\(P_{desorb}\)): This parameter behaved very
similarly to the standard STPs, increasing steadily with P fertilization
and confirming its role as a ``capacity'' indicator. * \textbf{Rate
Constant (}\(k\)): The rate constant showed a more complex pattern, with
no strong, consistent trend with fertilization. This suggests that while
fertilization increases the \emph{amount} of available P, it may not
change the intrinsic \emph{release rate}. * \textbf{Initial P Flux
(}\(J_0\)): As the product of \(P_{desorb}\) and \(k\), this parameter
integrates both capacity and intensity. It showed a strong positive
response to fertilization, driven primarily by the increase in
\(P_{desorb}\).

These initial observations suggest that the kinetic parameters,
particularly the rate constant \(k\), may provide unique information
about the soil's P dynamics not captured by static tests alone. The next
section will use formal statistical models to test these relationships.

\subsection{Predicting P Parameters from Soil
Properties}\label{sec-p-params-soil-props}

To understand the underlying drivers of the standard and kinetic P
parameters, and to test \textbf{Hypotheses 1b and 2b}, a series of
linear mixed-effects models were fitted. Each model predicted one of the
P parameters based on the core soil properties: organic carbon
(\(C_{org}\)), clay content, silt content, pH, and
dithionite-extractable Al (\(Al_d\)) and Fe (\(Fe_d\)). The results are
summarized in Table~\ref{tbl-soil-prop-models}.

\begin{longtable}[]{@{}
  >{\raggedright\arraybackslash}p{(\linewidth - 10\tabcolsep) * \real{0.2222}}
  >{\raggedright\arraybackslash}p{(\linewidth - 10\tabcolsep) * \real{0.1806}}
  >{\raggedright\arraybackslash}p{(\linewidth - 10\tabcolsep) * \real{0.1389}}
  >{\raggedright\arraybackslash}p{(\linewidth - 10\tabcolsep) * \real{0.1389}}
  >{\raggedright\arraybackslash}p{(\linewidth - 10\tabcolsep) * \real{0.1528}}
  >{\raggedright\arraybackslash}p{(\linewidth - 10\tabcolsep) * \real{0.1667}}@{}}

\caption{\label{tbl-soil-prop-models}Results of linear mixed-effects
models predicting P parameters from intrinsic soil properties.
Significance codes: `\emph{\textbf{' p \textless{} 0.001, '}' p
\textless{} 0.01, '}' p \textless{} 0.05.}

\tabularnewline

\toprule\noalign{}
\begin{minipage}[b]{\linewidth}\raggedright
Predictor/Model
\end{minipage} & \begin{minipage}[b]{\linewidth}\raggedright
\(P_{desorb}\)
\end{minipage} & \begin{minipage}[b]{\linewidth}\raggedright
\(k\)
\end{minipage} & \begin{minipage}[b]{\linewidth}\raggedright
\(J_0\)
\end{minipage} & \begin{minipage}[b]{\linewidth}\raggedright
\(P_{CO_2}\)
\end{minipage} & \begin{minipage}[b]{\linewidth}\raggedright
\(P_{AAE10}\)
\end{minipage} \\
\midrule\noalign{}
\endhead
\bottomrule\noalign{}
\endlastfoot
Intercept & 21.444 & 0.454 & 21.189 & 14.014 & 23.126 \\
\(Al_{d}\) & -8.706*** & -0.072*** & -8.631*** & -4.417*** &
-9.473*** \\
\(Fe_{d}\) & -1.068*** & 0.005 & -1.084*** & -0.845*** & -0.606*** \\
Clay & -0.006*** & -0.016*** & -0.085*** & 0.015 & -0.029*** \\
\(C_{org}\) & 0.612 & 0.137 & 1.250 & 0.269 & 1.454 \\
pH & -0.018*** & -0.021*** & -0.092*** & 0.124 & 0.012 \\
Silt & -0.000*** & 0.004 & 0.008 & -0.015*** & -0.049*** \\
\(R^2_m\) & 0.393 & 0.212 & 0.368 & 0.362 & 0.494 \\
\(R^2_c\) & 0.996 & 0.915 & 0.993 & 0.995 & 0.997 \\

\end{longtable}

The analysis reveals that the capacity-based P pools and the kinetic
rate constant are controlled by different sets of soil properties,
strongly supporting the hypotheses.

In line with \textbf{Hypothesis 2b}, the kinetic capacity parameter,
\textbf{Desorbable P (\(P_{desorb}\))}, showed a highly significant
negative relationship with both dithionite-extractable iron (\(Fe_d\))
and aluminum (\(Al_d\)). This provides strong evidence that the total
pool of free oxides, which represent the primary P sorption surfaces in
the soil, is a key factor controlling the size of the readily desorbable
P pool.

Also confirming \textbf{Hypothesis 2b}, the \textbf{Rate Constant
(\emph{k})} was governed by a different set of properties. It was not
significantly influenced by the free oxides but instead showed a
significant negative relationship with \texttt{Clay} content and a
positive relationship with organic carbon (\(C_{org}\)). This clearly
distinguishes the kinetic component from the capacity component,
suggesting that while oxides control \emph{how much} P can be held, soil
texture and organic matter influence \emph{how fast} it can be released.

The standard STP methods showed patterns consistent with
\textbf{Hypothesis 1b}. \textbf{Organic Carbon (\(C_{org}\))} had a
highly significant positive effect on \(P_{AAE10}\), and \textbf{pH} had
a significant negative effect, as predicted. The relationship of the STP
measures with the dithionite-extractable oxides was less consistent than
that of \(P_{desorb}\), with only \(P_{AAE10}\) showing a significant
negative link to \(Al_d\).

\subsection{Predictive Modeling of Agronomic
Outcomes}\label{sec-agronomic-modeling}

To formally evaluate the predictive power of the standard STP methods
against the kinetic parameters, a series of linear mixed-effects models
were fitted for each of the primary agronomic response variables.

\subsubsection{\texorpdfstring{Predicting Site-Normalized Yield
(\(Y_{norm}\))}{Predicting Site-Normalized Yield (Y\_\{norm\})}}\label{predicting-site-normalized-yield-y_norm}

When predicting yield normalized to the site's own potential, the
standard STP methods, particularly \(P_{CO_2}\), were the most effective
predictors (Table~\ref{tbl-ynorm-models}). The model including both STP
methods (\(P_{CO_2}*P_{AAE10}\)) achieved a marginal R² of 0.22,
explaining a substantial portion of the variance in within-site yield
response. The kinetic model also performed well, explaining 16.5\% of
the variance (marginal R² = 0.165), with the desorbable P pool
(\(P_{desorb}\)) being a highly significant predictor. This indicates
that for optimizing yield within a given field, both static and kinetic
capacity measures are effective.

\begin{longtable}[]{@{}
  >{\raggedright\arraybackslash}p{(\linewidth - 8\tabcolsep) * \real{0.3146}}
  >{\raggedright\arraybackslash}p{(\linewidth - 8\tabcolsep) * \real{0.1236}}
  >{\raggedright\arraybackslash}p{(\linewidth - 8\tabcolsep) * \real{0.1348}}
  >{\raggedright\arraybackslash}p{(\linewidth - 8\tabcolsep) * \real{0.2360}}
  >{\raggedright\arraybackslash}p{(\linewidth - 8\tabcolsep) * \real{0.1910}}@{}}

\caption{\label{tbl-ynorm-models}Results of linear mixed-effects models
predicting Site-Normalized Yield (\(Y_{norm}\)).}

\tabularnewline

\toprule\noalign{}
\begin{minipage}[b]{\linewidth}\raggedright
Predictor/Model
\end{minipage} & \begin{minipage}[b]{\linewidth}\raggedright
\(P_{CO_2}\)
\end{minipage} & \begin{minipage}[b]{\linewidth}\raggedright
\(P_{AAE10}\)
\end{minipage} & \begin{minipage}[b]{\linewidth}\raggedright
\(P_{CO_2}*P_{AAE10}\)
\end{minipage} & \begin{minipage}[b]{\linewidth}\raggedright
\(k * P_{desorb}\)
\end{minipage} \\
\midrule\noalign{}
\endhead
\bottomrule\noalign{}
\endlastfoot
Intercept & 1.059*** & 0.532*** & 1.096*** & 0.980 \\
\(k\) & & & & 2.262 \\
\(J_0\) & & & & 0.931 \\
\(P_{desorb}\) & & & & -0.063 \\
\(P_{AAE10}\) & & 0.120*** & -0.006 & \\
\(P_{CO_2}\) & 0.162*** & & 0.137 & \\
\(P_{CO_2} \times P_{AAE10}\) & & & 0.016 & \\
\(R^2_m\) & 0.218 & 0.198 & 0.220 & 0.014 \\
\(R^2_c\) & 0.358 & 0.474 & 0.365 & 0.360 \\

\end{longtable}

\subsubsection{\texorpdfstring{Predicting National-Normalized Yield
(\(Y_{rel}\))}{Predicting National-Normalized Yield (Y\_\{rel\})}}\label{predicting-national-normalized-yield-y_rel}

When predicting yield normalized to the national average, a different
pattern emerged (Table~\ref{tbl-yrel-models}). The kinetic model
(\(k * P_{desorb}\)) was the strongest predictor, achieving a marginal
R² of 0.12. Critically, the \textbf{rate constant (\emph{k})} and its
interaction with \(P_{desorb}\) (representing the initial flux \(J_0\))
were both significant. In contrast, the standard STP methods, while
still significant, explained less variance. This supports the hypothesis
that the \emph{speed} of P release (\emph{k}) becomes a more important
factor for predicting yield potential across diverse pedoclimatic
conditions.

\begin{longtable}[]{@{}
  >{\raggedright\arraybackslash}p{(\linewidth - 8\tabcolsep) * \real{0.3146}}
  >{\raggedright\arraybackslash}p{(\linewidth - 8\tabcolsep) * \real{0.1236}}
  >{\raggedright\arraybackslash}p{(\linewidth - 8\tabcolsep) * \real{0.1348}}
  >{\raggedright\arraybackslash}p{(\linewidth - 8\tabcolsep) * \real{0.2360}}
  >{\raggedright\arraybackslash}p{(\linewidth - 8\tabcolsep) * \real{0.1910}}@{}}

\caption{\label{tbl-yrel-models}Results of linear mixed-effects models
predicting National-Normalized Yield (\(Y_{rel}\)).}

\tabularnewline

\toprule\noalign{}
\begin{minipage}[b]{\linewidth}\raggedright
Predictor/Model
\end{minipage} & \begin{minipage}[b]{\linewidth}\raggedright
\(P_{CO_2}\)
\end{minipage} & \begin{minipage}[b]{\linewidth}\raggedright
\(P_{AAE10}\)
\end{minipage} & \begin{minipage}[b]{\linewidth}\raggedright
\(P_{CO_2}*P_{AAE10}\)
\end{minipage} & \begin{minipage}[b]{\linewidth}\raggedright
\(k * P_{desorb}\)
\end{minipage} \\
\midrule\noalign{}
\endhead
\bottomrule\noalign{}
\endlastfoot
Intercept & 104.862*** & 75.343*** & 130.274*** & 56.375 \\
\(k\) & & & & 377.498** \\
\(J_0\) & & & & 171.507** \\
\(P_{desorb}\) & & & & -27.486* \\
\(P_{AAE10}\) & & 7.111** & -6.537 & \\
\(P_{CO_2}\) & 8.853** & & 23.091 & \\
\(P_{CO_2} \times P_{AAE10}\) & & & -3.110 & \\
\(R^2_m\) & 0.074 & 0.063 & 0.078 & 0.022 \\
\(R^2_c\) & 0.569 & 0.537 & 0.596 & 0.439 \\

\end{longtable}

\subsubsection{\texorpdfstring{Predicting P-Uptake
(\(P_{up}\))}{Predicting P-Uptake (P\_\{up\})}}\label{predicting-p-uptake-p_up}

For predicting P-Uptake, both the standard STP methods and the kinetic
model performed well, explaining a similar amount of variance
(Table~\ref{tbl-pexport-models}). The model combining both standard
tests (\(P_{CO_2}*P_{AAE10}\)) had the highest marginal R² (0.07). This
suggests that for predicting the total amount of P a crop will acquire,
measures of the P pool size (capacity) are robust and sufficient.

\begin{longtable}[]{@{}
  >{\raggedright\arraybackslash}p{(\linewidth - 8\tabcolsep) * \real{0.3146}}
  >{\raggedright\arraybackslash}p{(\linewidth - 8\tabcolsep) * \real{0.1236}}
  >{\raggedright\arraybackslash}p{(\linewidth - 8\tabcolsep) * \real{0.1348}}
  >{\raggedright\arraybackslash}p{(\linewidth - 8\tabcolsep) * \real{0.2360}}
  >{\raggedright\arraybackslash}p{(\linewidth - 8\tabcolsep) * \real{0.1910}}@{}}

\caption{\label{tbl-pexport-models}Results of linear mixed-effects
models predicting P-Export (\(P_{up}\)).}

\tabularnewline

\toprule\noalign{}
\begin{minipage}[b]{\linewidth}\raggedright
Predictor/Model
\end{minipage} & \begin{minipage}[b]{\linewidth}\raggedright
\(P_{CO_2}\)
\end{minipage} & \begin{minipage}[b]{\linewidth}\raggedright
\(P_{AAE10}\)
\end{minipage} & \begin{minipage}[b]{\linewidth}\raggedright
\(P_{CO_2}*P_{AAE10}\)
\end{minipage} & \begin{minipage}[b]{\linewidth}\raggedright
\(k * P_{desorb}\)
\end{minipage} \\
\midrule\noalign{}
\endhead
\bottomrule\noalign{}
\endlastfoot
Intercept & 27.522*** & 8.090 & 30.632* & 29.599*** \\
\(k\) & & & & 22.622 \\
\(J_0\) & & & & 11.928 \\
\(P_{desorb}\) & & & & 1.954 \\
\(P_{AAE10}\) & & 4.824*** & -0.805 & \\
\(P_{CO_2}\) & 5.177*** & & 8.069 & \\
\(P_{CO_2} \times P_{AAE10}\) & & & -0.814 & \\
\(R^2_m\) & 0.064 & 0.073 & 0.065 & 0.064 \\
\(R^2_c\) & 0.625 & 0.603 & 0.623 & 0.648 \\

\end{longtable}

\subsubsection{\texorpdfstring{Predicting P-Balance
(\(P_{bal}\))}{Predicting P-Balance (P\_\{bal\})}}\label{predicting-p-balance-p_bal}

The most striking result was found when predicting the P-Balance
(Table~\ref{tbl-pbalance-models}). In stark contrast to the standard STP
methods, which showed no significant ability to predict the P-Balance,
the kinetic model was a powerful predictor. The kinetic model explained
\textbf{57\% of the variance} in P-Balance (marginal R² = 0.572), with
the \textbf{Desorbable P pool (\(P_{desorb}\))} being the dominant,
highly significant predictor. This indicates that the \(P_{desorb}\)
parameter from the kinetic experiment is a vastly superior measure of
the soil's P budget and its response to long-term fertilization compared
to standard STP tests.

\begin{longtable}[]{@{}
  >{\raggedright\arraybackslash}p{(\linewidth - 8\tabcolsep) * \real{0.3146}}
  >{\raggedright\arraybackslash}p{(\linewidth - 8\tabcolsep) * \real{0.1236}}
  >{\raggedright\arraybackslash}p{(\linewidth - 8\tabcolsep) * \real{0.1348}}
  >{\raggedright\arraybackslash}p{(\linewidth - 8\tabcolsep) * \real{0.2360}}
  >{\raggedright\arraybackslash}p{(\linewidth - 8\tabcolsep) * \real{0.1910}}@{}}

\caption{\label{tbl-pbalance-models}Results of linear mixed-effects
models predicting P-Balance (\(P_{bal}\)).}

\tabularnewline

\toprule\noalign{}
\begin{minipage}[b]{\linewidth}\raggedright
Predictor/Model
\end{minipage} & \begin{minipage}[b]{\linewidth}\raggedright
\(P_{CO_2}\)
\end{minipage} & \begin{minipage}[b]{\linewidth}\raggedright
\(P_{AAE10}\)
\end{minipage} & \begin{minipage}[b]{\linewidth}\raggedright
\(P_{CO_2}*P_{AAE10}\)
\end{minipage} & \begin{minipage}[b]{\linewidth}\raggedright
\(k * P_{desorb}\)
\end{minipage} \\
\midrule\noalign{}
\endhead
\bottomrule\noalign{}
\endlastfoot
Intercept & 4.441 & 7.691 & 3.649 & 43.833*** \\
\(k\) & & & & 84.993 \\
\(J_0\) & & & & 33.029 \\
\(P_{desorb}\) & & & & 16.947*** \\
\(P_{AAE10}\) & & -0.794 & 0.187 & \\
\(P_{CO_2}\) & -0.928 & & -2.442 & \\
\(P_{CO_2} \times P_{AAE10}\) & & & 0.462 & \\
\(R^2_m\) & 0.001 & 0.001 & 0.001 & 0.572 \\
\(R^2_c\) & 0.810 & 0.807 & 0.811 & 0.744 \\

\end{longtable}

\section{Discussion}\label{discussion}

\section{Conclusion}\label{conclusion}

\section{Acknowledgments}\label{acknowledgments}

\section{Legal Disclosure}\label{legal-disclosure}

\newpage

\section{References}\label{references}

\phantomsection\label{refs}
\begin{CSLReferences}{1}{0}
\bibitem[\citeproctext]{ref-Abelson1999Dilemma}
Abelson, P. H. (1999). The global phosphorus dilemma: A new insight into
an old problem. \emph{Science}, \emph{283}(5410), 2015.
\url{https://doi.org/10.1126/science.283.5410.2015}

\bibitem[\citeproctext]{ref-R-lme4}
Bates, D., Mächler, M., Bolker, B., \& Walker, S. (2015). Fitting linear
mixed-effects models using {lme4}. \emph{Journal of Statistical
Software}, \emph{67}(1), 1--48.
\url{https://doi.org/10.18637/jss.v067.i01}

\bibitem[\citeproctext]{ref-Bell2013Factors}
Bell, R. W., Bell, M. J., \& M. Tavora, W. J. G. de. (2013). Factors
influencing the soil test calibration for colwell p and wheat under
winter-dominant rainfall. \emph{Crop and Pasture Science}, \emph{64}(5),
489--501. \url{https://doi.org/10.1071/CP13019}

\bibitem[\citeproctext]{ref-Berg2019Biochemistry}
Berg, J. M., Tymoczko, J. L., Jr., G. J. G., \& Stryer, L. (2019).
\emph{Biochemistry} (9th ed.). W. H. Freeman; Company.

\bibitem[\citeproctext]{ref-Bohn2002SoilWater}
Bohn, H. L., Myer, R. A., \& O'Connor, G. A. (2002). \emph{Soil and
water chemistry: An integrative approach} (3rd ed.). John Wiley \& Sons.

\bibitem[\citeproctext]{ref-Brady2016Soils}
Brady, N. C., \& Weil, R. R. (2016). \emph{The nature and properties of
soils} (15th ed.). Pearson.

\bibitem[\citeproctext]{ref-Fardeau1991Phosphate}
Fardeau, J. C., Morel, C., \& Boniface, R. (1991). Phosphate ion
transfer from soil to soil solution: Kinetic parameters.
\emph{Agronomie}, \emph{11}(9), 787--797.
\url{https://doi.org/10.1051/agro:19910908}

\bibitem[\citeproctext]{ref-flossmannExtractionMethodCharacterizing1982}
Flossmann, R., \& Richter, D. (1982). \emph{Extraction method for
characterizing the kinetics of phosphorus release from solid soil to
soil solution.}

\bibitem[\citeproctext]{ref-FAL1996Methodenbuch}
Forschungsanstalt für Agrarökologie und Landbau (FAL). (1996).
\emph{Methodenbuch für boden-, pflanzen- und nährstoffanalysen}. FAL.

\bibitem[\citeproctext]{ref-Frossard2000Processes}
Frossard, E., Condron, L. M., Oberson, A., Sinaj, S., \& Fardeau, J. C.
(2000). Processes governing phosphorus availability in temperate soils.
\emph{Journal of Environmental Quality}, \emph{29}(1), 15--23.
\url{https://doi.org/10.2134/jeq2000.00472425002900010003x}

\bibitem[\citeproctext]{ref-Gerke2010Humic}
Gerke, J. (2010). Humic (organic matter)-al(fe)-phosphate complexes: An
underestimated phosphate form in soils and source of plant-available
phosphate. \emph{Soil Science}, \emph{175}(9), 417--425.
\url{https://doi.org/10.1097/SS.0b013e3181f26a1d}

\bibitem[\citeproctext]{ref-Hinsinger2001Phosphorus}
Hinsinger, P. (2001). Phosphorus dynamics in the soil-plant continuum: A
review. \emph{Plant and Soil}, \emph{237}(2), 167--191.
\url{https://doi.org/10.1023/A:1013339317511}

\bibitem[\citeproctext]{ref-Hirte2018Relationship}
Hirte, J., Leifeld, J., Laggoun-Défarge, A., Mayer, P., \& Gubler, J. M.
(2018). Relationship between soil phosphorus, phosphorus budget, and
soil properties in swiss agricultural soils. \emph{Ambio},
\emph{47}(Suppl 1), 53--64.
\url{https://doi.org/10.1007/s13280-017-0987-9}

\bibitem[\citeproctext]{ref-hirteYieldResponseSoil2021}
Hirte, J., Richner, W., Orth, B., Liebisch, F., \& Flisch, R. (2021).
Yield response to soil test phosphorus in {Switzerland}: {Pedoclimatic}
drivers of critical concentrations for optimal crop yields using
multilevel modelling. \emph{Science of The Total Environment},
\emph{755}, 143453.
\url{https://doi.org/10.1016/j.scitotenv.2020.143453}

\bibitem[\citeproctext]{ref-Hirte2021Yield}
Hirte, J., Stüssel, C. E. M., Leifeld, J., Gubler, J. M., Sinaj, S., \&
Frossard, E. (2021). Yield response to soil test phosphorus in
switzerland: Pedoclimatic drivers of critical concentrations for optimal
crop yields using multilevel modelling. \emph{Agriculture, Ecosystems \&
Environment}, \emph{309}, 107270.
\url{https://doi.org/10.1016/j.agee.2020.107270}

\bibitem[\citeproctext]{ref-Holford1997SoilP}
Holford, I. C. R. (1997). Soil phosphorus: Its measurement, and its
uptake by plants. \emph{Australian Journal of Soil Research},
\emph{35}(2), 227--239. \url{https://doi.org/10.1071/S96047}

\bibitem[\citeproctext]{ref-Johnston2001Phosphorus}
Johnston, A. E., Poulton, P. R., \& Goulding, K. W. T. (2001).
Phosphorus in soils, crop production and water quality. \emph{The
Scientific World Journal}, \emph{1}, 304--311.
\url{https://doi.org/10.1100/tsw.2001.272}

\bibitem[\citeproctext]{ref-Kuang2012Phosphorus}
Kuang, W., Wang, W. J., Liu, X. J., Cui, Y. F., Chen, Z. H., Wang, B.
R., \& Lin, X. Y. (2012). Phosphorus desorption kinetics in soils with
different long-term fertilization: A comparison of kinetic models.
\emph{Journal of Soils and Sediments}, \emph{12}, 739--749.
\url{https://doi.org/10.1007/s11368-012-0498-8}

\bibitem[\citeproctext]{ref-R-lmerTest}
Kuznetsova, A., Brockhoff, P. B., \& Christensen, R. H. B. (2017).
lmerTest package: Tests in linear mixed effects models. \emph{Journal of
Statistical Software}, \emph{82}(13), 1--26.
\url{https://doi.org/10.18637/jss.v082.i13}

\bibitem[\citeproctext]{ref-R-mlr3}
Lang, M., Binder, M., Richter, J., Schratz, P., Casalicchio, G., Coors,
S., Pfisterer, F., Fischer, S., Au, Q., \& Bischl, B. (2019). {mlr3}: A
modern object-oriented machine learning framework in {R}. \emph{Journal
of Open Source Software}, \emph{4}(44), 1903.
\url{https://doi.org/10.21105/joss.01903}

\bibitem[\citeproctext]{ref-McDowell2001Approximating}
McDowell, R. W., \& Sharpley, A. N. (2001). Approximating phosphorus
release from soils to surface runoff and subsurface drainage.
\emph{Journal of Environmental Quality}, \emph{30}(2), 508--520.
\url{https://doi.org/10.2134/jeq2001.302508x}

\bibitem[\citeproctext]{ref-Mehra1960Iron}
Mehra, O. P., \& Jackson, M. L. (1960). Iron oxide removal from soils
and clays by a dithionite-citrate system buffered with sodium
bicarbonate. \emph{Clays and Clay Minerals}, \emph{7}, 317--327.

\bibitem[\citeproctext]{ref-NIH2023Phosphorus}
National Institutes of Health, Office of Dietary Supplements. (2023).
\emph{Phosphorus: Fact sheet for health professionals}.
\url{https://ods.od.nih.gov/factsheets/Phosphorus-HealthProfessional/}

\bibitem[\citeproctext]{ref-Nelson2021Lehninger}
Nelson, D. L., Cox, M. M., \& Hoskins, A. A. (2021). \emph{Lehninger
principles of biochemistry} (8th ed.). Macmillan Learning.

\bibitem[\citeproctext]{ref-Nye2000Solute}
Nye, P. H., \& Tinker, P. B. (2000). \emph{Solute movement in the
rhizosphere}. Oxford University Press.

\bibitem[\citeproctext]{ref-R-nlme}
Pinheiro, J., Bates, D., DebRoy, S., Sarkar, D., \& R Core Team. (2022).
\emph{Nlme: Linear and nonlinear mixed effects models}.
\url{https://CRAN.R-project.org/package=nlme}

\bibitem[\citeproctext]{ref-R-base}
R Core Team. (2022). \emph{R: A language and environment for statistical
computing}. R Foundation for Statistical Computing.
\url{https://www.R-project.org/}

\bibitem[\citeproctext]{ref-Rast1996Eutrophication}
Rast, W., \& Thornton, J. A. (1996). A eutrophication of waters: Control
and management. In G. E. Likens (Ed.), \emph{Limnology and oceanography}
(pp. 253--289). Chapman \& Hall.

\bibitem[\citeproctext]{ref-Rowe2016LegacyP}
Rowe, H., Withers, P. J. A., Baas, P., Chan, N. P., Doody, M., Holiman,
J., Jacobs, B., Li, H., MacKay, R. T. E. M., O'Dwyer, M. D. C. P.,
O'Keeffe, M. G., O'Rourke, G. F. S. F. T. G. T., Rafferty, S. M., \&
Carton, O. T. (2016). Integrating legacy soil phosphorus into
sustainable nutrient management strategies for future food, bioenergy
and water security. \emph{Nutrient Cycling in Agroecosystems},
\emph{104}, 393--412. \url{https://doi.org/10.1007/s10705-015-9726-1}

\bibitem[\citeproctext]{ref-Schofield1955Potassium}
Schofield, R. K. (1955). Can a precise meaning be given to 'available'
soil phosphorus? \emph{Soils and Fertilizers}, \emph{18}, 373--375.

\bibitem[\citeproctext]{ref-Sharpley2000Phosphorus}
Sharpley, A. N., Daniel, T. C., Sims, J. T., Lemunyon, J. L., Stevens,
R. G., \& Parry, R. W. (2000). Agricultural phosphorus and
eutrophication. \emph{Journal of Environmental Quality}, \emph{29}(1),
1--9. \url{https://doi.org/10.2134/jeq2000.00472425002900010001x}

\bibitem[\citeproctext]{ref-Sharpley2003Water}
Sharpley, A. N., Duiker, S. W., \& Feyereisen, G. W. (2003). Improving
the agricultural water quality in the u.s. \emph{Journal of
Environmental Quality}, \emph{32}(2), 421--439.
\url{https://doi.org/10.2134/jeq2003.4210}

\bibitem[\citeproctext]{ref-Sims2005Phosphorus}
Sims, J. T., \& Sharpley, A. N. (Eds.). (2005). \emph{Phosphorus:
Agriculture and the environment}. American Society of Agronomy, Crop
Science Society of America, Soil Science Society of America.
\url{https://doi.org/10.2134/agronmonogr46}

\bibitem[\citeproctext]{ref-Sparks2003Environmental}
Sparks, D. L. (2003). \emph{Environmental soil chemistry} (2nd ed.).
Academic Press. \url{https://doi.org/10.1016/B978-0-12-656446-4.50001-X}

\bibitem[\citeproctext]{ref-Sposito2008Chemistry}
Sposito, G. (2008). \emph{The chemistry of soils} (2nd ed.). Oxford
University Press.

\bibitem[\citeproctext]{ref-Stevenson1994Humus}
Stevenson, F. J. (1994). \emph{Humus chemistry: Genesis, composition,
reactions} (2nd ed.). John Wiley \& Sons.

\bibitem[\citeproctext]{ref-VanVeldhoven1987Malachite}
Van Veldhoven, P. P., \& Mannaerts, G. P. (1987). Inorganic and organic
phosphorus in the scheldt estuary. \emph{Estuarine, Coastal and Shelf
Science}, \emph{25}(6), 755--765.

\bibitem[\citeproctext]{ref-VDLUFA2000Methodenbuch}
Verband Deutscher Landwirtschaftlicher Untersuchungs- und
Forschungsanstalten (VDLUFA). (2000). \emph{Methodenbuch band i: Die
untersuchung von böden}. VDLUFA-Verlag.

\end{CSLReferences}

\section{Appendix}\label{appendix}

\section{Supplements}\label{supplements}



\newpage
\phantomsection
\addcontentsline{toc}{section}{\listfigurename}
\listoffigures
\newpage
\phantomsection
\addcontentsline{toc}{section}{\listtablename}
\listoftables


\end{document}
